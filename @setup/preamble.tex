% !TeX spellcheck = en_US
% !TeX encoding = ISO-8859-1

%#
%# ------------------------------------------------------------------------------
%#
%#    Copyright (C) 2014  Nuno AJ de Aniceto <nuno.aja@gmail.com>
%#
%#    This program is free software: you can redistribute it and/or modify
%#    it under the terms of the GNU General Public License as published by
%#    the Free Software Foundation, either version 3 of the License, or
%#    (at your option) any later version.
%#
%#    This program is distributed in the hope that it will be useful,
%#    but WITHOUT ANY WARRANTY; without even the implied warranty of
%#    MERCHANTABILITY or FITNESS FOR A PARTICULAR PURPOSE.  See the
%#    GNU General Public License for more details.
%#
%#    You should have received a copy of the GNU General Public License
%#    along with this program.  If not, see <http://www.gnu.org/licenses/>.
%#
%# ------------------------------------------------------------------------------
%#
%# If you have the interest, feel free to improve the script, contact me !
%#

% - - - - - - - - - - - - - - - - - - - - - - - - - - - - - - - - - - - - -
%
% 		package and document setups...
%
%

\documentclass[a4paper]{article}
%\documentclass[conference]{IEEEtran}

%\usepackage[cm]{fullpage}
\usepackage[margin=2.5cm]{geometry}
\usepackage{lipsum, blindtext}
\usepackage{texments, listings, minted}
\usepackage{fixltx2e, setspace, hyperref, framed, wrapfig, empheq}
\usepackage{graphicx, color, xcolor, fancyhdr, multicol}
\usepackage{braket, mathtools, amsmath, amsfonts, amssymb, mathtools}
\usepackage{array, booktabs, arydshln}
%\usepackage{arrayjob} % conflicts with array
\usepackage{tikz}
\usetikzlibrary{positioning, shapes}

% input & language internationalization
\usepackage[latin9]{inputenc}
%\usepackage[utf8]{inputenc}
\usepackage[T1]{fontenc}
\usepackage{eurosym}\def\texteuro{\euro}
\usepackage[english, portuguese]{babel}
