% !TeX spellcheck = en_US
% !TeX encoding = ISO-8859-1

%#
%# ------------------------------------------------------------------------------
%#
%#    Copyright (C) 2014  Nuno AJ de Aniceto <nuno.aja@gmail.com>
%#
%#    This program is free software: you can redistribute it and/or modify
%#    it under the terms of the GNU General Public License as published by
%#    the Free Software Foundation, either version 3 of the License, or
%#    (at your option) any later version.
%#
%#    This program is distributed in the hope that it will be useful,
%#    but WITHOUT ANY WARRANTY; without even the implied warranty of
%#    MERCHANTABILITY or FITNESS FOR A PARTICULAR PURPOSE.  See the
%#    GNU General Public License for more details.
%#
%#    You should have received a copy of the GNU General Public License
%#    along with this program.  If not, see <http://www.gnu.org/licenses/>.
%#
%# ------------------------------------------------------------------------------
%#
%# If you have the interest, feel free to improve the script, contact me !
%#

% - - - - - - - - - - - - - - - - - - - - - - - - - - - - - - - - - - - - -
%
% 		commands and environment setups...
%
%

% -- colors --
\definecolor{ist-blue}{HTML}{009DE0}
\definecolor{ist-gray}{HTML}{46555F}
\definecolor{answer-blue}{HTML}{009DE0}
\definecolor{answer-gray}{HTML}{46555F}


% -- matrix with different line-spacing --

\makeatletter
\renewcommand*\env@matrix[1][\arraystretch]{% \arraystretch
\edef\arraystretch{#1}%
	\hskip -\arraycolsep
	\let\@ifnextchar\new@ifnextchar
	\array{*\c@MaxMatrixCols c}}
\makeatother

% -- matrices and vectors --

%\begingroup
%\renewcommand*{\arraystretch}{1.5}
% (pmatrix expression here)
%\endgroup

% 2/3d column-vectors \cvec[a]{b}{c} or \cvec{x}{y}
\newcommand*\cvec[3][]{
    \begin{pmatrix}\ifx\relax#1\relax\else#1\\\fi#2\\#3\end{pmatrix}
}

% -- boxes --

\newenvironment{bb}{\empheq[box=\boxed]{flalign*}{\textbf{Answer}}\\[10pt]} {\endempheq{\\[10pt]}}


% -- inline boxing an important equation (counter) --

\newcommand*\textfbox[2][equation title]{%
  \begin{tabular}[b]{@{}c@{}}#1\\\fbox{#2}\end{tabular}}

% -- usage example --
% \begin{empheq}[box={\textfbox[A different equation title]}]{align}
%   important equation #x (with counter)
% \end{empheq}


% -- avoid indentation (on itemizations, enumerations, listings, etc) --
% problem: does not allow spanning over several pages - to be resolved !
\newcommand\NoIndent[1]{
  \par\vbox{\parbox[t]{\linewidth}{#1}}
}

% -- answering to question with boxes --

%\makeatletter
%\newcommand{\answer}[1][(...answer here...) \\ \vspace{25pt}]{
%	\NoIndent{
%		\begin{center}
%				%%\begin{minipage}[c][4em][s]{\textwidth}
%				\begin{minipage}
%				{\dimexpr\linewidth-5\fboxsep-1\fboxrule\relax}
%					\centering{\textbf{\color{answer-blue}{Answer}}}\\[5pt]
%					\flushleft{\color{answer-gray}{#1}}
%				\end{minipage}
%		\end{center}
%	}
%	\vspace{25pt}
%}

%\makeatletter
\newcommand\answer[1][\par (...answer here...) \vs]{
	\noindent
	{
		\par
		\begin{center}
%			\centering{\textbf{\color{answer-blue}{Resposta}}}\\
			\flushleft{\color{answer-gray}{#1}}
		\end{center}
%		\flushleft
%		\par
	}
}
\newcommand\answerBox[1][(...answer here...) \\ \vspace{25pt}]{
	\noindent
	{
		\par
		\begin{center}
			\flushleft{\color{answer-gray}{#1}}
		\end{center}
		\par
	}
}

\makeatletter
\newcommand{\answerNI}[1][(...answer here...) \\ \vspace{25pt}]{
%	\begin{framed}
	\noindent
	{
		\begin{center}
			\centering{\textbf{\color{answer-blue}{Answer}}}\\
			\flushleft{\color{answer-gray}{#1}}
		\end{center}
	}
	\vspace{15pt}
%	\end{framed}
}

\makeatletter
\newcommand{\solution}[1][(...solution here...) \\ \vspace{25pt}]{
	%\begin{framed}
	\NoIndent
	{
		\begin{center}
			\centering{\textbf{\color{answer-blue}{Solution}}}\\
			\flushleft{\color{answer-gray}{{#1}}}
		\end{center}
	}
	\vspace{15pt}
	%\end{framed}
}


\newcommand\homework[1]{
	\begin{document}
	\maketitle \thispagestyle{empty}
	{#1}
	\end{document}
}

% % % % % % % % % % % % % % % % % % % % % % % % % % % % % % % % % % % % % %
%
% 	to test font types and set other options
%
%
%\makeatletter
%\newcommand*{\alphabet}{abcdefghijklmnopqrstuvwxyz}
%\newcommand*{\showalphabetwidth}[2]{%
%  \fontfamily{#1}\selectfont
%  \settowidth{\@tempdima}{\alphabet}%
%  \alphabet~-- width for #2 at 1\@ptsize pt: \the\@tempdima
%}
%\makeatother
%
%\showalphabetwidth{cmr}{Computer Modern}
%\showalphabetwidth{ptm}{Times New Roman}
%\showalphabetwidth{ppl}{Palatino}


%minipage:
%\hangindent=-2in
%\hangafter=5
%\usepackage{parskip}
%\setlength{\parindent}{5pt}
%
% % % % % % % % % % % % % % % % % % % % % % % % % % % % % % % % % % % % % %

% - - - - - - - - - - - - - - - - - - - - - - - - - - - - - - - - - - - - -
%
% 		document info and style setup:
%	 		title, author, date, headers, etc...
%
%


%\makeatletter
%\title{
%	\begin{center}
%		%\includegraphics[scale=0.25]{ist_logo.png}\\
%		\includegraphics[scale=0.4]{../@logos/ist_logo.png}\\
%		\vspace{20pt}
%		\hrulefill\\
%		\vspace{15pt}
%		\vspace{-100pt}
%		\textbf{Information and Computation for AI}\\
%		Homework \#7\\
%	\end{center}
%		\vspace{15pt}

\renewcommand*{\familydefault}{\sfdefault}

\makeatletter
\title{
		\vspace{-2cm}
		\begin{multicols}{2}
			\flushright
			\color{ist-gray}
			\textbf{\huge{\discipline}}\\{\Large \subTitle}\\
		\columnbreak
			\flushleft
			\includegraphics[scale=0.35]{\logotype}
		\end{multicols}
		\vspace{-1cm}
} \let\Title\@title

\date{
	\today \\
	\hrulefill\\
	\vspace{15pt}
} \let\Date\@date

%\fontfamily{ppl}\selectfont
%\renewcommand*{\familydefault}{\rmdefault}
%\renewcommand*{\familydefault}{\sfdefault}

\everymath{\displaystyle} 
\pagestyle{fancy}
\hypersetup{colorlinks=true, urlcolor=blue, unicode=true, linkcolor=blue}

\fancyhf{} % sets both header and footer to nothing
%\renewcommand{\headrulewidth}{0pt}


%\lhead{\authorName{} - \authorId}
\lhead{Grupo \groupId}
\chead{\includegraphics[scale=0.075]{\logotype}}
\rhead{p�gina $\thepage$}

\pagenumbering{arabic}

\newcommand{\vs}{\vspace{12pt}}
\newcommand{\hs}{\hspace{12pt}}
\parskip 		= 12pt
\baselineskip 	= 12pt

% define SQL script
%\begin{minted}[mathescape, linenos, numbersep=5pt, tabsize=4, xleftmargin=10pt,
%               xrightmargin=10pt, gobble=1, frame=single, framesep=2mm]{sql}
%	use AdventureWorksDW2012
%	
%	SELECT
%		SUM(SalesAmount)
%	FROM
%		FactInternetSales
%\end{minted}


%
%
% - - - - - - - - - - - - - - - - - - - - - - - - - - - - - - - - - - - - -
